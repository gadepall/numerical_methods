\documentclass[journal,12pt,twocolumn]{IEEEtran}
%
\usepackage{setspace}
\usepackage{multicol}
\usepackage{gensymb}
\usepackage{enumerate}
\usepackage{xcolor}
\usepackage{caption}
%\usepackage{subcaption}
%\doublespacing
\singlespacing
%\usepackage{epstopdf}
%\usepackage{graphicx}
%\usepackage{amssymb}
%\usepackage{relsize}
\usepackage[cmex10]{amsmath}
\usepackage{mathtools}
%\usepackage{amsthm}
%\interdisplaylinepenalty=2500
%\savesymbol{iint}
%\usepackage{txfonts}
%\restoresymbol{TXF}{iint}
%\usepackage{wasysym}
\usepackage{amsthm}
\usepackage{mathrsfs}
\usepackage{txfonts}
\usepackage{stfloats}
\usepackage{cite}
\usepackage{cases}
\usepackage{subfig}
%\usepackage{xtab}
\usepackage{longtable}
\usepackage{multirow}
%\usepackage{algorithm}
%\usepackage{algpseudocode}
%\usepackage{enumitem}
\usepackage{mathtools}
\usepackage{iithtlc}
%\usepackage[framemethod=tikz]{mdframed}
\usepackage{listings}
\usepackage{amsmath}
\usepackage{polynomial}
\usepackage{url}
\def\UrlBreaks{\do\/\do-}
%\usepackage{stmaryrd}

       \usepackage[latin1]{inputenc}
       \usepackage{fullpage}
       \usepackage{color}
       \usepackage{array}
       \usepackage{longtable}
       \usepackage{calc}
       \usepackage{multirow}
       \usepackage{hhline}
       \usepackage{ifthen}

 \def\inputGnumericTable{}

%\usepackage{wasysym}
%\newcounter{MYtempeqncnt}
\DeclareMathOperator*{\Res}{Res}
%\renewcommand{\baselinestretch}{2}
\renewcommand\thesection{\arabic{section}}
\renewcommand\thesubsection{\thesection.\arabic{subsection}}
\renewcommand\thesubsubsection{\thesubsection.\arabic{subsubsection}}

\renewcommand\thesectiondis{\arabic{section}}
\renewcommand\thesubsectiondis{\thesectiondis.\arabic{subsection}}
\renewcommand\thesubsubsectiondis{\thesubsectiondis.\arabic{subsubsection}}

% correct bad hyphenation here
\hyphenation{op-tical net-works semi-conduc-tor}

\lstset{
language=Python,
frame=single, 
breaklines=true
}

%\lstset{
	%%basicstyle=\small\ttfamily\bfseries,
	%%numberstyle=\small\ttfamily,
	%language=python,
	%backgroundcolor=\color{white},
	%%frame=single,
	%%keywordstyle=\bfseries,
	%%breaklines=true,
	%%showstringspaces=false,
	%%xleftmargin=-10mm,
	%%aboveskip=-1mm,
	%%belowskip=0mm
%}

%\surroundwithmdframed[width=\columnwidth]{lstlisting}


\begin{document}
%

\theoremstyle{definition}
\newtheorem{theorem}{Theorem}[section]
%\newtheorem{problem}{Problem}[section]
\newtheorem{problem}{Problem}
\newtheorem{proposition}{Proposition}
%\newtheorem{proposition}{Proposition}[section]
\newtheorem{lemma}{Lemma}[section]
\newtheorem{corollary}[theorem]{Corollary}
\newtheorem{example}{Example}[section]
%\newtheorem{definition}{Definition}[section]
\newtheorem{definition}{Definition}
%\newtheorem{definition}{Definition}
%\newtheorem{algorithm}{Algorithm}[section]
%\newtheorem{cor}{Corollary}
\newcommand{\BEQA}{\begin{eqnarray}}
\newcommand{\EEQA}{\end{eqnarray}}
\newcommand{\define}{\stackrel{\triangle}{=}}

\bibliographystyle{IEEEtran}
%\bibliographystyle{ieeetr}

\providecommand{\nCr}[2]{\,^{#1}C_{#2}} % nCr
\providecommand{\nPr}[2]{\,^{#1}P_{#2}} % nPr
\providecommand{\mbf}{\mathbf}
\providecommand{\pr}[1]{\ensuremath{\Pr\left(#1\right)}}
\providecommand{\qfunc}[1]{\ensuremath{Q\left(#1\right)}}
\providecommand{\sbrak}[1]{\ensuremath{{}\left[#1\right]}}
\providecommand{\lsbrak}[1]{\ensuremath{{}\left[#1\right.}}
\providecommand{\rsbrak}[1]{\ensuremath{{}\left.#1\right]}}
\providecommand{\brak}[1]{\ensuremath{\left(#1\right)}}
\providecommand{\lbrak}[1]{\ensuremath{\left(#1\right.}}
\providecommand{\rbrak}[1]{\ensuremath{\left.#1\right)}}
\providecommand{\cbrak}[1]{\ensuremath{\left\{#1\right\}}}
\providecommand{\lcbrak}[1]{\ensuremath{\left\{#1\right.}}
\providecommand{\rcbrak}[1]{\ensuremath{\left.#1\right\}}}
\theoremstyle{remark}
\newtheorem{rem}{Remark}
\newcommand{\sgn}{\mathop{\mathrm{sgn}}}
\providecommand{\abs}[1]{\left\vert#1\right\vert}
\providecommand{\res}[1]{\Res\displaylimits_{#1}} 
\providecommand{\norm}[1]{\lVert#1\rVert}
\providecommand{\mtx}[1]{\mathbf{#1}}
\providecommand{\mean}[1]{E\left[ #1 \right]}
\providecommand{\fourier}{\overset{\mathcal{F}}{ \rightleftharpoons}}
%\providecommand{\hilbert}{\overset{\mathcal{H}}{ \rightleftharpoons}}
\providecommand{\system}{\overset{\mathcal{H}}{ \longleftrightarrow}}
	%\newcommand{\solution}[2]{\textbf{Solution:}{#1}}
\newcommand{\solution}{\noindent \textbf{Solution: }}
\providecommand{\dec}[2]{\ensuremath{\overset{#1}{\underset{#2}{\gtrless}}}}
%\numberwithin{equation}{section}
\numberwithin{equation}{problem}

%\numberwithin{problem}{subsection}
%\numberwithin{definition}{subsection}
\makeatletter
\@addtoreset{figure}{problem}
\makeatother

\let\StandardTheFigure\thefigure
%\renewcommand{\thefigure}{\theproblem.\arabic{figure}}
\renewcommand{\thefigure}{\theproblem}


%\numberwithin{figure}{subsection}

\def\putbox#1#2#3{\makebox[0in][l]{\makebox[#1][l]{}\raisebox{\baselineskip}[0in][0in]{\raisebox{#2}[0in][0in]{#3}}}}
     \def\rightbox#1{\makebox[0in][r]{#1}}
     \def\centbox#1{\makebox[0in]{#1}}
     \def\topbox#1{\raisebox{-\baselineskip}[0in][0in]{#1}}
     \def\midbox#1{\raisebox{-0.5\baselineskip}[0in][0in]{#1}}

\vspace{3cm}

\title{ 
\logo{
Interpolation and Least Squares
}
}

\author{G V V Sharma$^{*}$ %<-this  stops a space
\thanks{*The author is with the Department
of Electrical Engineering, IIT, Hyderabad
502285 India e-mail: gadepall@iith.ac.in. All material in the manuscript is released under GNU GPL.  Free to use for all.}% <-this % stops a space
}



% make the title area
\maketitle

%\newpage

%\tableofcontents


\IEEEpeerreviewmaketitle

\bigskip

\begin{abstract}
Through examples, this manual introduces methods for polynomial interpolation and curve fitting like Newton, Lagrange's and
least squares methods. 
Python codes are provided for all these methods.
\end{abstract}
%
\section{Newton's Interpolation Formula}
\begin{theorem}
Newton's interpolation formula for a function $f(x)$ is given by \cite{newton, nptel}
\begin{equation}
\label{eq:newton}
f(x)=\sum _{k=0}^{\infty }\Delta ^{k}y \, {}^{u}C_k
\end{equation}
%
where
\begin{align}
{}^{u}C_k &=\frac{u(u-1)(u-2)\cdots (u-k+1)}{k!}
\\
\Delta ^{n}y&=\sum _{k=0}^{n}{\binom {n}{k}}(-1)^{n-k}y_k
\label{eq:diff_newton}
\\
u & = \frac{x-a}{h}
\label{eq:newton_u}
\end{align}
%
with $a$ as the initial point and $h$, the step-size.
\end{theorem}
%
%
\begin{problem}
For the table in Table \ref{table:newton}, 
obtain \eqref{eq:diff_newton} for $n = 0, 1, \dots 5$.

\end{problem}
\input{./figs/newton.tex}
\solution The following code generates the coefficients
\begin{equation}
\label{eq:newton_coeff}
1.121, 0.002, 0.0005, -0.0015, 0.002, -0.0025
\end{equation}
\lstinputlisting{./codes/newton_coeff.py}
%
\begin{problem}
Obtain the interpolating polynomial $P_{5}(u)$.
\end{problem}
%
\solution  From \eqref{eq:newton_coeff} and \eqref{eq:newton},
\begin{multline}
\label{eq:newton_poly}
P_5(u) = 1.121+ u \times .002
\\
+\frac{u(u-1)}{2}(.0005) + \frac{u(u-1)(u-2)}{3!}\times (-.0015)
\\
 +\frac{u(u-1)(u-2)(u-3)}{4!}(.002)+ 
 \\
 \frac{u(u-1)(u-2)(u-3)(u-4)}{5!}
\times(-.0025).
\end{multline}
\begin{problem}
\label{prob:newton}
Using \eqref{eq:newton_poly},
interpolate the value of the function at $ x = 0.0045.$
\end{problem}
\solution From \eqref{eq:newton_u},
\begin{equation}
u = \frac{x-a}{h} = \frac{0.0045-0}{0.001} = 4.5
\end{equation}
The following code results in
\begin{equation}
P_5(u) = 1.128400390625
\end{equation}
\lstinputlisting{./codes/newton.py}
%
\section{Lagrange's Interpolation Formula}
\begin{theorem}
Given a set of data points
\begin{equation*}
(x_{0},y_{0}),\ldots ,(x_{j},y_{j}),\ldots ,(x_{k},y_{k})
\end{equation*}
where no two $x_j$ are the same, \cite{lagrange}, 
\begin{equation}
\label{eq:lagrange}
f(x)=\sum _{j=0}^{k}y_{j}\prod _{\begin{smallmatrix}0\leq m\leq k\\m\neq j\end{smallmatrix}}{\frac {x-x_{m}}{x_{j}-x_{m}}}
%={\frac {(x-x_{0})}{(x_{j}-x_{0})}}\cdots {\frac {(x-x_{j-1})}{(x_{j}-x_{j-1})}}{\frac {(x-x_{j+1})}{(x_{j}-x_{j+1})}}\cdots {\frac {(x-x_{k})}{(x_{j}-x_{k})}}
\end{equation}
\end{theorem}
\begin{problem}
Using the following data \cite{nptel_lagrange}, find by \eqref{eq:lagrange}, the value of $ f(x)$ at $ x = 10$
\end{problem}
\input{./figs/lagrange.tex}
\solution  The following code gives
\begin{equation}
f(10) = 13.1978451179
\end{equation}
\lstinputlisting{./codes/lagrange.py}
\begin{problem}
Using  \eqref{eq:lagrange}, find $x$ if $f(x) = 16$. 
\end{problem}
\solution Interchange $x$ and $y$ in the previous code and evaluate f(16).
\section{Least Squares}
Use the following data for the following problems.
\begin{equation}
\label{eq:ls_data}
\brak{x_i,y_i} = \brak{0,5}, \brak{2,4}, \brak{4,1}, \brak{6,6}, \brak{8,7}.
\end{equation}
\begin{problem}
Obtain a matrix solution to find a line of best fit from the given data.
\end{problem}
\solution Let
\begin{equation}
\label{eq:ls_line}
y = mx + c
\end{equation}
be the equation of the line.  Using the data in \eqref{eq:ls_data} and \eqref{eq:ls_line} results in the matrix equation
\begin{equation}
\label{eq:ls_mat}
\mbf{A}\mbf{z} = \mbf{y}
\end{equation}
%
where 
%
\begin{equation}
A = 
\begin{pmatrix}
x_1 & 1
\\
x_2 & 1
\\
\vdots & \vdots
\\
x_5 & 1
\end{pmatrix},
\mbf{z} = 
\begin{pmatrix}
m
\\
c
\end{pmatrix},
\mbf{y} = 
\begin{pmatrix}
y_1 
\\
y_2 
\\
\vdots 
\\
y_5 
\end{pmatrix},
\end{equation}
%
The solution to \eqref{eq:ls_mat}  is obtained as
\begin{equation}
\label{eq:ls_line_sol}
\mbf{z} = \brak{\mbf{A}^{T}\mbf{A}}^{-1}\mbf{A}\mbf{y}
\end{equation}
\begin{problem}
Obtain $\mbf{z}$ in \eqref{eq:ls_line_sol} and sketch \eqref{eq:ls_line} using this information.
\end{problem}
\solution The following code results in
\begin{equation}
\mbf{z} =
\begin{pmatrix}
0.3
\\
3.4
\end{pmatrix} 
\end{equation}
%
and plots Fig. \ref{fig:ls_line}.
\lstinputlisting{./codes/ls_line.py}
\begin{figure}[!h]
\centering
\includegraphics[width=\columnwidth]{./figs/ls_line.eps}
\caption{Linear Estimate.}
\label{fig:ls_line}
\end{figure}

\begin{problem}
Obtain the estimate for $y = ax^2 + bx + c$.
\end{problem}
\solution The following code results in
\begin{equation}
\mbf{z} =
\begin{pmatrix}
 0.21428571
 \\
 -1.41428571
 \\
  5.11428571
\end{pmatrix} 
\end{equation}
%
and plots Fig. \ref{fig:ls_quad}.
\lstinputlisting{./codes/ls_quad.py}
\begin{figure}[!h]
\centering
\includegraphics[width=\columnwidth]{./figs/ls_quad.eps}
\caption{Quadratic Estimate.}
\label{fig:ls_quad}
\end{figure}
%
\begin{problem}
Write a program to find a curve of the form $y = ae^{bx}$ from the given
data.
\end{problem}
\solution Taking logarithms,
\begin{align}
\label{eq:ls_exp}
\ln y = \ln a + bx
\end{align}
\solution
Note that \eqref{eq:ls_exp} has the same form as \eqref{eq:ls_line}. The following code results in
\begin{equation}
\begin{pmatrix}
a
\\
b
\end{pmatrix}
= \begin{pmatrix}
1.05540067 \\ 1.13099846
\end{pmatrix} 
\end{equation}
and plots Fig. \ref{fig:ls_exp}.
\lstinputlisting{./codes/ls_exp.py}
\begin{figure}[!h]
\centering
\includegraphics[width=\columnwidth]{./figs/ls_exp.eps}
\caption{Exponential Estimate.}
\label{fig:ls_exp}
\end{figure}
%
\begin{problem}
Write a program to find a curve of the form $y = ax^{ b}$ from the given
data.
\end{problem}
\solution Taking logarithms,
\begin{align}
\label{eq:ls_pow}
\ln y = \ln a + b\ln x
\end{align}
\solution
Note that \eqref{eq:ls_pow} has the same form as \eqref{eq:ls_line}. The following code results in
\begin{equation}
\begin{pmatrix}
a
\\
b
\end{pmatrix}
= 
\begin{pmatrix}
 1.00450417 \\ 1.34195488
\end{pmatrix} 
\end{equation}
and plots Fig. \ref{fig:ls_pow}.
\lstinputlisting{./codes/ls_pow.py}
\begin{figure}[!h]
\centering
\includegraphics[width=\columnwidth]{./figs/ls_pow.eps}
\caption{Estimate.}
\label{fig:ls_pow}
\end{figure}
%

\bibliography{IEEEabrv,gvv_inter}
\end{document}


